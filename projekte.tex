\chapter{Projekte}
  \section{Netzwerk}
Aktuell sind in unserem Netzwerk X Menschen\footnote{Siehe Anhang Y - Liste der weXelwirker}. Das Wachstum ist gesund und stetig.
  \section{Wirtschaftliche Zusammenarbeit}
Größere Projekte. Mehr gemeinsam, auch in der Rechtsform.
Ziel für jeden einzelnen: Sicherheit, Beständigkeit, Versicherung/Altervorsorge.
    \subsection{Genossenschaft}
  \section{CoWorking}
    \subsection{Paderborn}
    \subsection{Wankheim}
weXelwirken Härten befindet sich in Kusterdingen-Wankheim "`Auf den Härten"' (so heißt das Landschaftsgebiet) - direkt zwischen Reutlingen und Tübingen.
%
Coworking (Zusammenarbeit) in ländlicher Gegend - ein innovatives Projekt.



CoWorking ist nicht "`billiges Arbeiten"', sondern nachhaltige Ressourcennutzung, Netzwerken und Kooperatives Handeln.




Du findest uns im Wankheimer Zentrum, Hauptstraße 16.
%Das folgende in Listen packen?
Unsere Ausstattung beeinhaltet: Internet, WLAN, LAN, Laserdrucker SW bis DIN A4, Tintenstrahldrucker Farbe bis DIN A4, Aktenvernichter, Schneidemaschine, etwas Büromaterial (Briefumschläge, Ordner, Locher, Tacker usw.), Tee- und Kaffeeküche, Abstellkammer, Kellerraum und interessierte Menschen.
%
In nächster Nähe findest Du Volksbank und Sparkasse, Bushaltestelle, Bäckereifiliale, Metzgerei, Hofladen, Druckerei, Fotograf und einiges mehr.



Wenn Du einen Büroplatz zum Arbeiten suchst, dann kannst Du diesen mieten.
%Liste?
Sprich mit uns, wir finden Deinen individuellen Tarif.\\
\\
    1. Schnuppertag: kostenlos\\
    Tagestarif: 15 Eur zzgl. MwSt.\\ 
    10-Tages-Pass: 140 Eur zzgl. MwSt.\\
    Monatstarif: 250 Eur zzgl. MwSt. (inkl. eigenem Schlüssel)\\
    Studententarif (mit Ausweis): 10 Eur pro Tag\\
\\
(jeweils inkl. Internet, Kaffee/Tee, FairUse der restlichen Ausstattung.)\\

Wir laden Dich ein, unseren Zusammenarbeitsraum zu testen.
%
Ruf an. Komm vorbei.
%
Beachte auch unsere Veranstaltungen unter Termine.
     \subsection{Halle Westfalen}
     \subsection{CoWorking Partner}
  Stuttgart, Würzburg
  \section{Radio weXelwirken}      
$http://wexelwirken.de/radio\_wexelwirken.html$
      


Zitat aus dem Wellenreiter, der Programmzeitschrift der Wüsten Welle Tübingen:



Zwischen Tübingen und Reutlingen liegen "`Die Härten"', ein Landschaftsgebiet mit vielen interessanten Menschen.
%
Wir berichten über die Härtenorte und interviewen die Härtemer.
%
Außerdem geben wir Computertipps und stellen Neues aus der Technikwelt vor.
%
Wir sind Lea, Sebastian, Joachim und Christopher.



Wir sind eine Radiosendung, die aus dem weXelwirken Härten in Wankheim entstanden ist.
%
Wir senden jeden zweiten Freitag (gerade Kalenderwoche) von 12-13 Uhr auf der Wüsten Welle.\\
96,6 MHz UKW über Antenne in Tübingen Reutlingen und im ganzen Sendegebiet\\
97,45 MHz im Kabelnetz von Tübingen und Reutlingen\\
101,15 MHz im Kabelnetz von Rottenburg\\
und natürlich im Internetlivestream\\
$http://wueste-welle.de:8000/stream.m3u$

\section{Computercafé}
Mit dem Veranstaltungsformat "'Computercafé"' wollen wir Computeranfänger, ältere Menschen und andere Neudigitale beim Umgang mit dem Werkzeug Computer unterstützen. Das Konzept baut auf freiwillige Mentoren, die individuell Fragen beantworten. Die Teilnehmer können so den sicheren Umgang mit dem Medium in ihrem eigenen Tempo erlernen. Durch die gewonnenen Kontakte soll mit der Zeit ein hilfreiches Netzwerk für alle entstehen. Bitte gebt die Informationen weiter an Menschen, die Fragen zum Computer haben oder die gerade anfangen, sich damit zu beschäftigen. Hilfsbereite Menschen, die als Mentoren helfen möchten, können sich gerne bei uns melden. Die Teilnahme ist kostenlos.
\section{Elektrischer Kamin}

\section{Härten-Blog}
Die Unterstützung der regionalen Infrastruktur ist ein wichtiges Anliegen.
%
Mit dem Härten-Blog haben wir eine zentrale Plattform für die Region Härten geschaffen.
%
Lokale Gewerbetreibende, Vereine, Künstler und Privatpersonen können hier Informationen einstellen und bekommen.
%
Das Ziel der Vernetzung wird zusätzlich mit "'Härten-Blog-Treffen"' verwirklicht.
%
Ergeben haben sich daraus schon mehrere Kontakte, die zu einigen Geschäften und guten Gesprächen geführt haben.
%
Jeder kann sich daran beteiligen und helfen, die Plattform aufzubauen und zu stärken.
\section{Unternehmen-OWL.de}
\section{KoKonsum}
Teilen statt Kaufen.
%###################################
MEHR TEXT

\section{Schlusssatz}
Sie baden gerade Ihr Gehirn darin!