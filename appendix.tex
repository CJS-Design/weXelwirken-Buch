\begin{appendix}
\addcontentsline{toc}{chapter}{Anhang}

\chapter{Entwicklungsgeschichte}
  \section{2012}
    \subsection{2. Workshop (Wankheim, 13-15.01.2012)}
Beim zweiten Workshop in Wankheim waren anwesend...
    \subsection{Wankheim}
      \subsubsection{Radio wexelwirken}
Neue Radiosendung beim Freien Radio Wüste Welle auf der 96,6 mit Tipps und Interviews von den Härten. Jede zweite Woche, Freitags, von 12-13 Uhr.
  \section{2011}
    \subsection{1. Workshop (Paderborn, 4.-6.11.2011)}
Beim ersten Workshop in Paderborn waren anwesend...
    \subsection{Vor den Workshops}
Auszug aus der Homepage, Text der sich aus der Steyerberg Klausur entwickelt hat und auf die Workshops und Vorträge hinarbeiten sollte.
\glqq 
Die Zukunft

weXelwirken will mehr sein als ein Arbeitsplatzvermieter. (Im folgenden sind immer beide Geschlechter gleichberechtigt gemeint, auch wenn eine weibliche oder männliche Form verwendet wird).
Die Problematiken

    In der Selbstständigkeit muss man vieles tun, was einem nicht liegt (z.B. will der Webdesigner nicht wirklich stundenlang über Steuern nachdenken, die Buchhalterin weiß nicht, wie sie sich ein Logo erstellt und der Programmierer kann nicht mit Kunden umgehen. Erweitere die Liste um Deine Erfahrungen.)
    Es entstehen Unsicherheiten bezüglich rechtlicher Fragen, bezüglich des Ablaufs in verschiedenen amtlichen Abläufen und in der Vorgehensweise bei verschiedenen Themen (z.B. Kundenakquise, Projektmanagement, steuerliche Abrechnungen, Risikoeinschätzung, Marketing und so weiter. Bei welchen Themen bist Du unsicher? In welchen Themen kennst Du Dich dafür bestens aus?)
    Die Informationssuche zu Themen, mit denen man sich nicht auskennt, nimmt viel Zeit in Anspruch (Wieviel Zeit verbringst Du mit Recherchen zu Themen, die Dich eigentlich gar nicht interessieren?)
    Durch verlorene Zeit leistest Du weniger als gedacht, hast weniger Spaß und verlierst die Motivation (Ich schätze, dass ich anfangs ca. 60\% meiner Zeit mit Dingen verbracht habe, die ich nicht tun wollte, aber tun musste. Wie ist/war das bei Dir?)
    Während ein Auftrag bearbeitet wird, fehlt die Zeit zur Akquise. Während der Akquise verbrauchst Du das Geld des letzten Auftrages. Der nächste Auftrag wird erst nach Fertigstellung bezahlt. Du musst kurze und lange Zeiträume finanziell überbrücken. (Wann akquirierst Du? Wann lebst Du? Gibt es hier Verbesserungsmöglichkeiten?)
    Viele vor Dir hatten die gleichen oder zumindest ähnliche Fragestellungen, Du kommst aber nicht an diese Menschen oder traust Dich nicht, diese zu fragen (Welche Antworten und Abläufe könntest Du Dir von wem abschauen? Wieviel Zeit würde das sparen?)
    Die rechtliche Verantwortung eines Selbstständigen ist sehr hoch und durch den Grundsatz "Unwissenheit schützt nicht vor Strafe" muss ein sehr hohes Wissen aufgebaut werden (Wo liegen rechtliche Stolpersteine für Dich? Kennst Du den Unterschied zwischen Werkvertag, Dienstvertag und Kaufvertrag? Unterschreiben Deine Kunden bei Dir einen Vertrag? Hast Du Verträge für Deine Subunternehmer? Hast Du einen Steuerberater? Beschäftigst Du regelmäßig einen Rechtsanwalt?)
    Selbstständige und Freiberufler verlieren Aufträge, weil Unternehmen die (vermeintliche?) Sicherheit einer Agentur mit mehreren Angestellten vorziehen. Bei großen Aufträgen sind Selbstständige oft von vornherein ungewünscht. (Hast Du schonmal einen Auftrag nicht bekommen, der dann einem Wettbewerber mit Angestellten zugesprochen wurde? Kennst Du die Gründe oder hast Du Vermutungen?)

Die Lösungen

    Netzwerke bilden, so dass ungewollte Themen ausgelagert werden können
    Netzwerke bilden, um Fragen klären zu können
    Mentor suchen, der einem beisteht und aus Erfahrung helfen kann
    Das Netzwerk nach außen darstellen, um auch größere Aufträge zu bekommen und den Auftraggebern Sicherheit bieten zu können

Die Idee

Wir haben eine Idee entwickelt, um die Problematiken anzugehen und die Arbeit für Selbstständige und Freiberufler neu zu gestalten. Im Mittelpunkt steht die Konzentration auf den eigenen Kernarbeitsbereich. Jeder Mensch bringt am meisten Leistung in dem Gebiet, in dem er sich auskennt und in dem er Spaß hat.

Diese Idee stellen wir ab Mitte September 2011 in Workshops und Vorträgen vor. Wir hoffen dabei auf konstruktives Feedback und viele Interessenten, die mit uns gemeinsam weiter an der Umsetzung arbeiten wollen. Wenn Du Dich dafür interessierst schreibe bitte an branko.canak@weXelwirken.net und christopher.schmidhofer@weXelwirken.net.

Gemeinsam mehr erreichen.
Danksagung

Wir bedanken uns bei allen, die bisher geholfen haben, gerade helfen und in Zukunft helfen werden, unsere Visionen umzusetzen und mitzutragen. Besonderer Dank gilt unserem Mentor Declan Kennedy.

\grqq
    \subsection{Klausur (Steyerberg, 4.-6.07.2011)}
Branko und Christopher, Mentor Declan Kennedy. Steyerberg Atmo

    \subsection{Paderborn}
    \subsection{Wankheim}
In Zusammenarbeit mit dem "`Härten Tag"' wurde zur regionalen Stärkung der "`Härten Blog"' ins Leben gerufen. Neben der Online-Adresse sind vor allem die regelmäßigen Offline-Treffen ein wichtiger Punkt in der Wahrnehmung von weXelwirken. Durch zahlreiche Veranstaltungen und eine gute Pressearbeit haben wir eine gute regionale Aufmerksamkeit erreicht.

Die Zahl der CoWorker wächst langsam, aber stetig. Dabei steht vor allem die gegenseitige Unterstützung im Vordergrund. Du findest hier immer eine hilfsbereite Person oder jemanden, der Dir den passenden Kontakt vermitteln kann.

Aktuell haben wir noch genügend freie Plätze für weitere CoWorker.

Wir haben viele Kontakte geknüpft zu ganz verschiedenen, sehr interessanten Menschen. Hier wurden Vorträge und Workshops zu den Themen Geldsystem, Nachhaltiges Marketing, Akquise, Erste Hilfe, Philosphie und Tanzen veranstaltet.
Die Teilnahme an regionalen Events wie der Gewerbeschau in Kusterdingen, der 900 Jahr Feier von Wankheim und den sehr erfolgreichen „Offenen Ateliers“ hat unseren Bekanntheitsgrad sehr gesteigert.
Durch die Vernetzung mit dem Härten-Tag ist früh der Härten-Blog entstanden. Auf dieser Plattform im Internet können alle Härtemer Veranstaltungen und Vereins- oder Unternehmensdaten eintragen, sowie Berichte schreiben. Beispielhaft nennen möchte ich hier Thomas Ellinger, ein lokaler Winzer, der jeden Monat aus dem Weinberg berichtet. Der Härten-Blog ist aber nicht nur online, sondern auch offline bei regelmäßigen Treffen ein zentraler Faktor der Vernetzung.
Ende September kam das Computercafè dazu. Die Unterstützung von älteren Menschen und Anfängern beim Umgang mit dem Werkzeug Computer liegt uns am Herzen. Ich glaube daran, dass dieses Werkzeug für jeden eine Bereicherung sein kann und niemand ausgeschlossen werden sollte oder sich selbst davon ausschließen sollte.
Es gäbe noch viel zu berichten. Zum Beispiel von den tatsächlichen Umsätzen, die sich hier ergeben haben. Die Gespräche in denen wir unterstützt haben oder unterstützt wurden. Aber das muss man meiner Meinung nach erlebt haben (und das können Sie erleben, wenn Sie sich hier einmieten.)

5. Kunstausstellungen in unseren Räumen – bereits ausgestellt haben
Karen Seekamp-Schnieder
Elsa Lunter
Petra Ohneseit
Gudrun Böhm
Martina Nehr Kley
  \section{2010}
    \subsection{Paderborn}
    \subsection{Wankheim}
Ein halbes Jahr nach dem Umzug auf die Härten fällt Christopher die Entscheidung, in Wankheim ein weXelwirken zu eröffnen. Die Absprache mit Branko ist schnell getroffen, ein Recht zur Verwendung der eingetragenen Marke wird übertragen.
Der Raum in Wankheim wird ab Januar 2011 angemietet. Schon ab Dezember 2010 kann der Raum bezogen werden. Christopher zieht hier für ein paar Tage einsam mit seinem Schreibtisch ein.

Die ersten Reaktionen aus der CoWorking Community beziehen sich vor allem auf geografische Fragen (Wo ist dieses "`Härten"'?). Die Bewohner der Ortschaften können mit dem begriff CoWorking-Space nichts anfangen. Viel Kraft wird auf die Erklärung verwendet, was wir eigentlich machen.
  \section{2009}
Unter dem dem Namen weXelwirken hat sich im Herbst 2009 in Paderborn eine Gruppe von jungen Selbständigen aus dem IT- und Medienbereich zusammengefunden. Die Zeit war reif für einen Aufbruch in eine neue Qualität des Arbeitens - raus aus der einsamen Enge des Homeoffices, rein in das professionellere Umfeld eines gemeinsamen Büros in bester Altstadt-Lage.

Nicht weil die Idee des CoWorking wünschen wert war, haben wir uns für diese Struktur entschieden. Umgekehrt wird ein Schuh draus: Den Notwendigkeiten und unseren Bedürfnissen folgend haben wir Strukturen geschaffen und fanden uns schließlich in etwas wieder, das man heute landläufig CoWorking-Space nennt.

\chapter{Geschichten}
  \section{Christophers erstes Mal}
Das erste Mal habe ich mit weXelwirken 2009 zu tun gehabt. Ich hatte beim Webmontag in Padebrorn einen Vortrag zum Thema Drupal gehalten. Daniel (acid) Schweighöfer war an diesem Webmontag dabei. Ein paar Wochen später flatterte bei Branko ein Projekt rein, dessen Umsetzung mit Drupal geplant war. Dabei kam dann von Daniel mein Name ins Spiel. Nach ein paar Telefonaten bin ich zum ersten mal zu weXelwirken gelaufen. Ich laufe also in das Büro rein. Dort sitzen 4 Personen. Zufällig kenne ich schon 3 davon, und nachdem ich diese begrüßt habe, sage ich zur vierten Person: Hallo, du bist der einzige den ich hier noch nicht kenne, du musst Branko sein. Das war der Beginn einer tollen \dots{} fruchtbaren Zusammenarbeit für ein Kunden-Projekt. 
Branko und alle Paderborner weXelwirker aus dieser Zeit haben mich unterstützt, Fragen zur Selbstständigkeit beantwortet und mir geholfen, als Einzelunternehmer erfolgreich zu sein. Dieses persönliche Erlebnis war so geil, die Zeit so intensiv und spannend, dass mich das noch heute trägt und motiviert. 
Diese Motivation, diese Energie, diese Art der gegenseitigen Hilfe vor zu leben und weiter zu geben, das ist seitdem ein Ziel für mich.

\chapter{Rede zum einjährigen Jubiläum von weXelwirken Wankheim}
1 Jahr weXelwirken in Wankheim - WOW

Sehr geehrte Damen und Herren, Hallo weXelwirker,

(meine Name ist Christopher Schmidhofer und) viele haben mich für verrückt erklärt, als ich sagte: Ich mache einen CoWorking-Space auf dem Land auf, in Wankheim auf den Härten.

Die Menschen, die nicht von hier sind, haben gefragt:
Wer oder was sind die Härten?

Den meisten konnte ich das mit "`Die Härten, das ist eine Hochebene zwischen Tübingen und Reutlingen"' erklären – einige konnten die geographische Richtung nur mit "`südlich von Stuttgart"' erahnen.

Die Menschen auf den Härten (sogenannte "`Härtemer"') konnten dagegen mit dem Wort "`CoWorking-Space"' nichts anfangen. Zusammenarbeitsplatz war da schon etwas verständlicher. Inzwischen werden wir als „Kommunikationszentrale“ bezeichnet, wie in der lokalen Presse nachzulesen ist. Das empfinden wir als großes Lob.
Erstaunt bin ich darüber, wie schnell die Idee hinter dem Begriff "`weXelwirken"' verinnerlicht wurde. Die Wechselwirkung zwischen Menschen, die Kooperation der Individuen, das wurde hier schnell richtig interpretiert... und umgesetzt.

Woher kommt dieser Name? 

weXelwirken, das ist eine Idee von Branko Canak. Er hat gemeinsam mit anderen das erste wexelwirken Büro in Paderborn gegründet. Ich habe in Paderborn studiert und mich 2009 selbstständig gemacht. Damals suchte Branko nach Partnern um ein größeres Projekt zu verwirklichen. Wir haben also telefoniert, einen Termin abgesprochen und ich bin das erstemal zu weXelwirken gelaufen. Ich laufe in das Büro rein. Dort sitzen 4 Personen. Zufällig kenne ich schon 3 davon, und nachdem ich diese begrüßt habe, sage ich zur vierten Person: Hallo, du bist der einzige den ich hier noch nicht kenne, du musst Branko sein. Das war der Beginn einer tollen … fruchtbaren Zusammenarbeit für ein Kunden-Projekt.

Branko und alle Paderborner weXelwirker aus dieser Zeit haben mich unterstützt, Fragen zur Selbstständigkeit beantwortet und mir geholfen, als Einzelunternehmer erfolgreich zu sein. Dieses persönliche Erlebnis war so        geil, die Zeit so intensiv und spannend, dass mich das noch heute trägt und motiviert. 
Diese Motivation, diese Energie, diese Art der gegenseitigen Hilfe vor zu leben und weiter zu geben, das ist seitdem ein Ziel für mich.

Nachdem wir aus beruflichen Gründen meiner damaligen Freundin, die inzwischen meine Verlobte ist, nach Wankheim gezogen sind, habe ich es im HomeOffice nicht lange ausgehalten – vor einem Jahr habe ich in Absprache mit Branko "`weXelwirken Härten"' eröffnet.

In diesem Jahr ist viel passiert.

Wir haben viele Kontakte geknüpft zu ganz verschiedenen, sehr interessanten Menschen. Hier wurden Vorträge und Workshops zu den Themen Geldsystem, Nachhaltiges Marketing, Akquise, Erste Hilfe, Philosphie und Tanzen veranstaltet.

Die Teilnahme an regionalen Events wie der Gewerbeschau in Kusterdingen, der 900 Jahr Feier von Wankheim und den sehr erfolgreichen "`Offenen Ateliers"' hat unseren Bekanntheitsgrad sehr gesteigert.

Durch die Vernetzung mit dem Härten-Tag ist früh der Härten-Blog entstanden. Auf dieser Plattform im Internet können alle Härtemer Veranstaltungen und Vereins- oder Unternehmensdaten eintragen, sowie Berichte schreiben. Beispielhaft nennen möchte ich hier Thomas Ellinger, ein lokaler Winzer, der jeden Monat aus dem Weinberg berichtet. Der Härten-Blog ist aber nicht nur online, sondern auch offline bei regelmäßigen Treffen ein zentraler Faktor der Vernetzung.

Ende September kam das Computercafè dazu. Die Unterstützung von älteren Menschen und Anfängern beim Umgang mit dem Werkzeug Computer liegt uns am Herzen. Ich glaube daran, dass dieses Werkzeug für jeden eine Bereicherung sein kann und niemand ausgeschlossen werden sollte oder sich selbst davon ausschließen sollte.

Heut war ausserdem die Premiere unserer neuen Radiosendung beim Freien Radio Wüste Welle auf der 96,6 mit Tipps und Interviews von den Härten. Jede zweite Woche, Freitags, von 12-13 Uhr.

Es gäbe noch viel zu berichten. Zum Beispiel von den tatsächlichen Umsätzen, die sich hier ergeben haben. Die Gespräche in denen wir unterstützt haben oder unterstützt wurden. Aber das muss man meiner Meinung nach erlebt haben (und das können Sie erleben, wenn Sie sich hier einmieten.)

Eine Sache habe ich jetzt noch nicht erwähnt:
Wir eröffnen heute bereits die 6. Kunstausstellung in unseren Räumen – bereits ausgestellt haben
\begin{itemize}
\item{Karen Seekamp-Schnieder}
\item{Elsa Lunter}
\item{Petra Ohneseit}
\item{Gudrun Böhm}
\item{Martina Nehr Kley}
\end{itemize}

und seit heute Heike Neudeck. Die heutige Ausstellung ist etwas besonderes für uns, weil es die erste Ausstellung ist, die in beiden weXelwirken-Räumlichkeiten stattgefunden haben wird, in Paderborn und in Wankheim. Ich behaupte jetzt mal ganz frech, es ist die erste Ausstellung überhaupt, die sowohl auf den Härten als auch in Paderborn gezeigt wurde. (Gegenthesen bitte später an mich.)

Nachdem ich die Ausstellung in Paderborn gesehen habe, hatte ich ein paar Tage lang riesigen Spaß daran, umher zuschauen und überall Gesichter zu entdecken. Stellen Sie sich ein Kind vor. Kinder entdecken überall etwas neues, die Welt ist spannend. Ich ging wie ein Kind durch Paderborn und fand neue Gesichter. Für ein paar Tage haben damals die mir bekannten Orte, die vertrauten Straßen und die oft gesehenen Ecken eine Neugierde weckende neue Dimension bekommen: Die Gesichter-Dimension. Für mich persönlich ist das eine Definition von Kunst: Eine neue Sicht auf die Dinge wecken, der Welt eine neue Dimension geben.

Ich bin gespannt darauf, ob auch Sie, so wie ich, in den nächsten Tagen Gesichter entdecken an unerwarteten Stellen. Im vorhin erwähnte Härten-Blog können Sie das dann gerne eintragen.

Ich bin fast am Ende meiner Worte hier. Ich möchte nun noch ein paar Personen Danke sagen.
Danke an Dagmar, für all die Unterstützung, das Verständnis und den Einsatz, um den Raum am Laufen zu halten. Ohne dich würde das alles nicht gehen.

Danke an Joachim, dafür dass du an die Idee glaubst und fast seit Anfang an hier dabei bist. Und für die vielen lustigen Nachtarbeitsrunden, sowie die unermüdliche Verbreitung des Konzepts. Danke auch an Joachims Familie, Petra und Lea, für Verständnis und Mithilfe. Die kleine Ausstellung da vorne am Schrank ist übrigens von Lea gemalt.

Danke an Sebastian, der neben seinem bilingualen Abi, Feuerwehr, Musikverein und vielen anderen Engagements hier noch Programmieren lernt, mit uns Radio macht, Mentor beim Computercafé ist und bei fast allen sonstigen Aktionen mit Engagement dabei ist. Petra, danke dir, dass du Ihn hier sein lässt.
Zeitweise läuft das nämlich so, dass Petra hier mit den Worten reinkommt: "`Ich bin hier vorbeigekommen, damit ich meinen Sohn auch mal sehe"'.

Danke an Gerlinde Münch für die Hilfe bei vorherigen, dieser und zukünftigen Ausstellungen. Danke an Diana für wunderbare Reden auf vergangenen und zukünftigen Vernissagen.
Danke an die Musiker, die bei unseren Vernissagen gespielt haben und natürlich auch an die, die heute spielen (übrigens auch zum zweiten Mal).

Danke auch an alle vorhin bereits genannten Künstlerinnen die hier mit viel Elan ausgestellt haben.
Besonders sei hier Karen Seekamp-Schnieder genannt, die uns bei einigen Veranstaltungen stark unterstützt hat.

Danke an Chris Kupi, für die stimmungsvollen Dekorationen im letzten Jahr. Wir bauen weiterhin auf dich.

Danke an Carmen Kuttler für deine Unterstützung und wirklich gute Gespräche.

Danke an Gaby Dannecker, Ryan Jones und alle sonstigen CoWorker des letzten und hoffentlich auch diesen Jahres.

Ein ganz besonderes Danke an Beate Simon, für immer neue Themen und Ideen, sowie breiter Unterstützung in vielen Gebieten.

Danke an Andreas Müller, Uwe Arndt, Aldo Palamides, Björn Starrach, Stefan Reiff, Hazelle Kurig, Dorothea Brandes, Alasdair MacAuley, Karin Ockert-Hoefler, Alfred Korte, Gudrun Witte-Borst, Cathy Eberle, Franzisca Schmidhofer, Jason-Luca Marnet und an alle anderen, die sich für uns eingesetzt haben, uns positive Rückmeldung gegeben haben, unsere Ideen verbreiten, an den Veranstaltungen teilgenommen, bei Veranstaltungen geholfen oder hier Veranstaltungen durchgeführt haben.  

Ein extra Danke an Dorothea für die Couch, den Glastisch und die Stühle, sowie das Leihen deines Klaviers.

Danke an die Feuerwehr Wankheim für das Leihen von Stühlen und Geschirr, sowie die geschenkte Spüle. Danke an die Lumpenkappelle "`Ammerheuler"' die morgen um 21 Uhr Ihren zweiten Live-Auftritt bei uns hat.

Danke auch an die lokale Presse für eine immer positive Berichterstattung.


Danke an alle Menschen, die uns unterstützt haben und das in Zukunft tun werden, die ich aber jetzt hier nicht genannt habe.

Das vorletzte Danke geht an die Paderborner Crew, für die Unterstützung und das tolle Netzwerk.  Besonderer Dank gilt hierbei meinem Freund Branko. Danke für deine Offenheit, für die langen Gespräche und für das gemeinsame Ausarbeiten einer Art, die Dinge zu tun.

Das letzte Danke, liebe gezwungene Zuhörer, geht an Euch für Eure Aufmerksamkeit.

Vielen herzlichen Dank für ein positives Jahr weXelwirken Wankheim, jetzt geht’s mit Schwung und Elan ins zweite Jahr.


\chapter{Vorstellung des Dunstkreises}
Die Vorstellung geschieht wertfrei in alphabetischer Reihenfolge nach dem Vornamen.
  \section{Branko Čanak}
\begin{itemize}
\item {Gründer des ersten weXelwirken in Paderborn (siehe Geschichte)}
\item {Teilnahme an Workshop 1}
\item {Teilnahme an Workshop 2}
\end{itemize}
  \section{Christopher Schmidhofer}
\glqq CoWorking ist für mich der Ausbruch aus dem Home Office, rein in ein aktives Netzwerk. Meine Produktivität steigt, ich werde kreativer. Meine Kunden kann ich in einem professionellen Umfeld empfangen. \grqq
\begin{itemize}
\item {Gründer des zweiten weXelwirken in Wankheim}
\item {Teilnahme an Workshop 1}
\item {Teilnahme an Workshop 2}
\end{itemize}

  \section{Dagmar Engels}
\glqq Die Trennung zwischen Privatem und Beruf ist für mich vorteilhaft. Nach anfänglicher Skepsis bin ich nun überzeugter CoWorker. \grqq
  \section{Elias Flügge}
\glqq CW ist gut, weil es mich in meiner Kreativität unterstützt und mir den sozialen Raum gibt, mich zu zeigen, wie ich bin. \grqq
  \section{Eugenia Allerdings}
\begin{itemize}
\item {Potenzielle Gründerin eines dritten weXelwirken}
\item {Teilnahme an Workshop 1}
\item {Teilnahme an Workshop 2}
\end{itemize}
  \section{Joachim Beyrowksi}
\glqq Die zusätzlichen Kontakte durch das Netzwerk sind sehr wertvoll. Voneinander zu lernen und sich gegenseitig voran zu bringen, das ist für mich wichtig beim CoWorking. \grqq
  \section{Vanessa Herrmann}
\glqq Unsere Gemeinschaft fördert die Motivation und Produktivität. Man kommt mit Menschen unterschiedlichster Professionen zusammen und erlangt einen guten Einblick in andere Berufsfelder. Dabei kann es auch vorkommen, dass ich Talente und Interessen an mir entdecke, auf die ich sonst nie gekommen wäre. \grqq

\section{Daniel Schweighöfer}
Serveradmin, Ruby on Rails
Berlin

\section{Viola Hellmuth}
Soziale Arbeit
Paderborn

\section{Carsten Birkelbach}
Java und Projektmanagement
Paderborn

\section{Conrad Berhörster}
generative Musikproduktion
Paderborn

\section{Chantal Pannacci}
Lebenskünstlerin, Social Media
Luxemburg

\section{Christina Schneider}
Webredaktion, Recherche
Paderborn

\section{Dejan Puric}
Tech Support
Paderborn

\section{Gaby Dannecker}
Versicherungen
Härten

\section{Andreas Hell}
Innovationsmanager
Bonn


\section{Elisabeth Jacob}
PHP, Datenbanken
Berlin


\section{Martin A. Gails}
EDV-Engineering


\section{Oliver G. Hoffmann}
Imagefilmen \& Produktvideos
Paris, Ludwigshafen

\section{Johannes Lintner}
Student, IT Security, Drupal
Paderborn

\section{Jolanthe Osiejuk}
Corporate Design
Paderborn

\section{Johanna Tovar}
Innenarchitektin
Detmold

\section{Katharina Rompf}
Drupal
Würzburg

\section{Diana Köster-Kande}
Kunstbesprechung, Künstlerbetreuung
Härten

\section{Luca Hammer}
Neue Arbeit
Paderborn

\section{Michael Rinker}
Öffentlichkeitsarbeit
Paderborn

\section{Rafael Schniedermann}
Programmierung
Paderborn

\section{Robert Rentz}
Web
Paderborn

\section{Ryan Mac Jones}
Startup: http://www.weand.co/
Härten, World

\section{Sebastian Ohneseit}
Schüler, Web, VIEL
Härten

\section{Tim Schmidt}
Java
eigtl. Paderborn

\section{Vanessa Herrmann}
Sozial
Paderborn

\section{Wolfgang Schreiber}
Wiki, Soziale Arbeit
Bielefeld

\chapter{Begriffsdefinitionen}
  \section{CoWork(er/ing) vs. weXelwirk(er/en)}

\end{appendix}