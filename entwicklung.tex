
\chapter{Gegenwart und Zukunft}
  \section{Aktueller Stand}
Aktueller Stand bezieht sich auf das Erstellungsdatum dieses Dokuments.
    \subsection{Netzwerk}
Aktuell sind in unserem Netzwerk X Menschen\footnote{Siehe Anhang Y - Liste der weXelwirker}. Das Wachstum ist gesund und stetig.
    \subsection{Paderborn}
    \subsection{Wankheim}
      \subsubsection{CoWorking}
weXelwirken Härten befindet sich in Kusterdingen-Wankheim "`Auf den Härten"' (so heißt das Landschaftsgebiet) - direkt zwischen Reutlingen und Tübingen.
%
Coworking (Zusammenarbeit) in ländlicher Gegend - ein innovatives Projekt.



CoWorking ist nicht "`billiges Arbeiten"', sondern nachhaltige Ressourcennutzung, Netzwerken und Kooperatives Handeln.




Du findest uns im Wankheimer Zentrum, Hauptstraße 16.
%Das folgende in Listen packen?
Unsere Ausstattung beeinhaltet: Internet, WLAN, LAN, Laserdrucker SW bis DIN A4, Tintenstrahldrucker Farbe bis DIN A4, Aktenvernichter, Schneidemaschine, etwas Büromaterial (Briefumschläge, Ordner, Locher, Tacker usw.), Tee- und Kaffeeküche, Abstellkammer, Kellerraum und interessierte Menschen.
%
In nächster Nähe findest Du Volksbank und Sparkasse, Bushaltestelle, Bäckereifiliale, Metzgerei, Hofladen, Druckerei, Fotograf und einiges mehr.



Wenn Du einen Büroplatz zum Arbeiten suchst, dann kannst Du diesen mieten.
%Liste?
Sprich mit uns, wir finden Deinen individuellen Tarif.\\
\\
    1. Schnuppertag: kostenlos\\
    Tagestarif: 15 Eur zzgl. MwSt.\\ 
    10-Tages-Pass: 140 Eur zzgl. MwSt.\\
    Monatstarif: 250 Eur zzgl. MwSt. (inkl. eigenem Schlüssel)\\
    Studententarif (mit Ausweis): 10 Eur pro Tag\\
\\
(jeweils inkl. Internet, Kaffee/Tee, FairUse der restlichen Ausstattung.)\\

Wir laden Dich ein, unseren Zusammenarbeitsraum zu testen.
%
Ruf an. Komm vorbei.
%
Beachte auch unsere Veranstaltungen unter Termine.
      \subsubsection{Radio weXelwirken}
      
$http://wexelwirken.de/radio\_wexelwirken.html$
      


Zitat aus dem Wellenreiter, der Programmzeitschrift der Wüsten Welle Tübingen:



Zwischen Tübingen und Reutlingen liegen "`Die Härten"', ein Landschaftsgebiet mit vielen interessanten Menschen.
%
Wir berichten über die Härtenorte und interviewen die Härtemer.
%
Außerdem geben wir Computertipps und stellen Neues aus der Technikwelt vor.
%
Wir sind Lea, Sebastian, Joachim und Christopher.



Wir sind eine Radiosendung, die aus dem weXelwirken Härten in Wankheim entstanden ist.
%
Wir senden jeden zweiten Freitag (gerade Kalenderwoche) von 12-13 Uhr auf der Wüsten Welle.\\
96,6 MHz UKW über Antenne in Tübingen Reutlingen und im ganzen Sendegebiet\\
97,45 MHz im Kabelnetz von Tübingen und Reutlingen\\
101,15 MHz im Kabelnetz von Rottenburg\\
und natürlich im Internetlivestream\\
$http://wueste-welle.de:8000/stream.m3u$
  \subsection{Halle-Westfalen}
  \subsection{CoWorking Partner}
  Stuttgart, Würzburg
  \section{mittelfristige Zukunft}
    \subsection{Verbreitung, Außenauftritt}
Barcamps und ähnliches sponsern.
    \subsection{Rechtsform}
      \subsubsection{CJS}
Eine Vereinsform für weXelwirken wäre in mehreren Punkten vorteilhaft.
%
Im Moment haben wir einige Menschen im Dunstkreis von weXelwirken, die mitgestalten und mitdiskutieren, aber nicht CoWorker sind.
%
Einige dieser Menschen sind historisch verbunden, aber inzwischen an anderen geographischen Orten unterwegs.
%
Einige sind regelmäßig an Veranstaltungen anwesend.
%
Diese Menschen einzubinden und mit weXelwirken enger zu verbinden, dazu wäre eine Vereinsform passend.
%
Nach extern gesehen ergebn sich für den Verein verschiedene Möglichkeiten, Kosten zu sparen.
%
In den Amtsblättern dürfen Vereine meist Veranstaltungen und ähnliches kostenlos ankündigen, Gemeindeeinrichtungen dürfen kostenlos oder vergünstigt mitbenutzt werden.
%
Die Räume könnten als Vereinsstätten deklariert und verwaltet werden (genauere Infos einholen!)
%
Diese Idee wurde auch von unserem Mentor Declan Kennedy aufgebracht.



Damit wäre eine Basis für die Verwaltung der Veranstaltungen, der Räumlichkeiten geschaffen, sowie eine Bindung an weXelwirken auch für Nicht CoWorker geschaffen.



Die gewünschte wirtschaftliche Kooperation benötigt eine eigene Form.
%
Hier ist der Gedanke an eine Genossenschaft immer noch verlockend.
%
Durch die Trennung in zwei eigenständig agierende Rechtsformen kann die Verschiedenheit der Ausrichtung, durch die gemeinsame Basis (wie in diesem Buch beschrieben) die Verknüpfung dargestellt werden.
%
Die Genossenschaft hat als einzige Rechtsform die Möglichkeit der flexiblen und dynamischen Mitgliederverwaltung.
%
Die meisten anderen Rechtsformen benötigen dafür Änderungen an Gesellschafterverträgen oder mehr Aufwand.
%
Die Frage, ob wir eine gemeinsame Rechtsform benötigen (Notwendigkeit) oder einfach nur schick finden, ist hier die falsche.
%
Es kommt darauf an, was die Interessenten an einer gemeinsamen Rechtsform interessant genug finden, den entsprechenden Aufwand da rein zu stecken.
%
Hier ist sicherlich noch Diskussionbedarf.
%
Es bleibt festzuhalten, dass in einer solchen gemeinsamen Rechtsform sicherlich nicht alle weXelwirker vertreten sein werden.
  \section{10 Jahres Vision}
100 weXelwirken Standorte international und über 1000 weXelwirker, die gemeinsam an einer begeisterten Menschheit bauen.
