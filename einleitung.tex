\chapter{Einleitung}
\glqq
Die Atmosphäre in einer auf sozialer Ebene geführten Gemeinschaft, in der man sich frei entfalten kann, ermöglicht es konstant und temporär Entwicklungsprozesse und Projekte interdisziplinär umzusetzen. 
\grqq \footnote{http://www.joachimbeyrowski.de vom 07.02.2012, geändert wurde das Wort "`Kundenprojekte"' in das allgemeinere "`Projekte"'.}

  \section{Warum dieses Buch?}
    weXelwirken ist eine Idee.
    %
    Eine Idee die sich in unserem Leben als notwendig erweist. 
    %
    Mit weXelwirken wollen wir Werkzeuge beschreiben, Werkzeuge für ein selbstverantwortliches Leben.
    %
    Aus unserer alltäglichen Erfahrung heraus sind wir zu der Überzeugung gelangt, dass selbstverantwortliches Leben in einer starken Gemeinschaft erstrebenswert ist.
    %
    Für uns persönlich nicht nur erstrebenswert, sondern existenziell notwendig.
    %
    Dabei ist weXelwirken keine Utopie, sondern eine bereits lebendige Art, die Dinge zu tun.
    %
    Wir leben so, und nun wird es Zeit dies nach außen zu tragen.
    %
    Dazu soll dieses Buch dienen.
    %
    Eine Annäherung an dieses Thema ist mit diesem Buch möglich.
    %
    Ein Verständnis wird sich aber nur durch persönliches Erleben ergeben.



    Wem das zu abgehoben ist: Wir sind eine offene Gruppe von engagierten Menschen, die selbstständig leben und selbstverantwortlich handeln wollen.
    %
    Konkret arbeiten wir wirtschaftlich zusammen, wo es passt.
    %
    Wir denken gemeinsam nach, über die in unserer Lebenswelt wichtigen Themen.
    %
    Wir unterstützen uns, wenn es notwendig wird.



    Dieses Buch bedarf einer ständigen Überarbeitung, eine neue Version wird jedes Jahr erscheinen.
    %
    Zwischendurch sind die aktuellen Texte und Ideen auf der Homepage zu finden: http://weXelwirken.net/buch
    %
    Bitte schauen Sie unter\\ http://weXelwirken.net/buch nach, ob Sie die aktuelle Version lesen.
    %
    Im Moment lesen Sie die Version 0.1 vom 13.04.2012 - laden Sie bitte die aktuellste Ausgabe herunter.
    %
    Wenn Sie Anregungen für Verbesserungen oder ähnliches haben, schreiben Sie an buch@weXelwirken.net .
  \section{Zielgruppe}
    Die Zielgruppe des Dokuments sind vor allem Menschen, die bereits in Kontakt mit weXelwirken stehen.
    %
    Die Zusammenfassung der Entwicklung wird uns allen dabei helfen, die richtigen Schritte zu machen. 




    Die besondere weXelwirken Atmosphäre lässt sich zwar mit Worten beschreiben, ist aber etwas, das gefühlt werden muss.
    %
    Dies lässt sich nur im persönlichen Kontakt erfahren.

  \section{weXelwirken in wenigen Minuten}
    Was ist unser Antrieb für weXelwirken? Warum verwenden wir Zeit darauf, eine solche Struktur zu beschreiben?



    Branko sagt: Ich erzähle vielen Menschen von der weXelwirken Art, weil ich glücklich danach lebe.
    %
    Eugenia sagt: Ich baue das Netzwerk aus und finde Stärke in Qualität und Quantität, die ich alleine nicht habe.
    %
    Christopher sagt: Mir hat weXelwirken so viel gebracht, das gebe ich weiter.



    Jeder hat seine eigene Geschichte, wie er mit weXelwirken in Kontakt gekommen ist. 
    %
    Dadurch ergeben sich auch verschiedene Punkte, die dem einzelnen an weXelwirken wichtig sind.
    %
    Diese unterschiedliche Erfahrung ergibt zusammengenommen weXelwirken.
    %
    Eine Definition oder auch eine Beschreibung von weXelwirken lässt daher immer viele Aspekte aus.
    %
    Die Kernaspekte sind das Netzwerk, also alle mit weXelwirken verbundenen Personen.
    %
    Hier geht es um wirtschaftliche Zusammenarbeit und um die Diskussion Themen aus unserer Lebenswelt.
    %
    Gerade die wirtschaftliche Zusammenarbeit ist dabei ein Kernthema, an dem wir arbeiten.
    %
    Wie können Selbstständige, Freiberufler und Andere zusammenarbeiten (Prozesse, Vertrauen, Haftung, ... )?
    %
    Ein weiterer Kernaspekt sind die Projekte, worunter auch das CoWorking fällt. 
    %
    Unsere Gemeinschaftsbüros sind derzeit für viele die erste Kontaktbasis zu weXelwirken.
    %
    Bei Veranstaltungen wie Vernissagen, Computercafé und vielen anderen werden Kontakte geknüpft und vertieft.
    %
    In diesem Buch lesen sie einen Teil dessen, was weXelwirken ausmacht.
    %
    Verstehen können Sie es aber mit persönlicher Erfahrung der weXelwirken Art, die Dinge zu tun.
    