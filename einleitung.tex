\chapter{Einleitung}
\glqq
Die Atmosphäre in einer auf sozialer Ebene geführten Gemeinschaft, in der man sich frei entfalten kann, ermöglicht es konstant und temporär Entwicklungsprozesse und Projekte interdisziplinär umzusetzen. 
\grqq \footnote{http://www.joachimbeyrowski.de vom 07.02.2012, geändert wurde das Wort "`Kundenprojekte"' in das allgemeinere "`Projekte"'.}

  \section{Warum dieses Buch?}
weXelwirken ist eine Idee.
%
Eine Idee die sich in unserem Leben als notwendig erweist. 
%
Mit weXelwirken wollen wir Werkzeuge beschreiben, Werkzeuge für ein selbstverantwortliches Leben.
%
Aus unserer alltäglichen Erfahrung heraus sind wir zu der Überzeugung gelangt, dass selbstverantwortliches Leben in einer starken Gemeinschaft erstrebenswert ist.
%
Für uns persönlich nicht nur erstrebenswert, sondern existenziell notwendig.
%
Dabei ist weXelwirken keine Utopie, sondern eine bereits lebendige Art, die Dinge zu tun.
%
Wir leben so, und nun wird es Zeit dies nach außen zu tragen.
%
Dazu soll dieses Buch dienen.
%
Eine Annäherung an dieses Thema ist mit diesem Buch möglich.
%
Ein Verständnis wird sich aber nur durch persönliches Erleben ergeben.



Wem das zu abgehoben ist: Wir sind eine offene Gruppe von engagierten Menschen, die selbstständig leben und selbstverantwortlich handeln wollen.
%
Konkret arbeiten wir wirtschaftlich zusammen, wo es passt.
%
Wir denken gemeinsam nach, über die in unserer Lebenswelt wichtigen Themen.
%
Wir unterstützen uns, wenn es notwendig wird.
  \section{Zielgruppe}
Die Zielgruppe des Dokuments sind vor allem Menschen, die bereits in Kontakt mit weXelwirken stehen.
%
Die Zusammenfassung der Entwicklung wird uns allen dabei helfen, die richtigen Schritte zu machen. 




Die besondere weXelwirken Atmosphäre lässt sich zwar mit Worten beschreiben, ist aber etwas, das gefühlt werden muss.
%
Dies lässt sich nur im persönlichen Kontakt erfahren.

  \section{weXelwirken in wenigen Minute}

Warum beschäftigen wir uns mit diesen Dingen? Was ist unser Antrieb?


Nach dieser Einleitung folgt ein Kapitel über die Ziele, also die Erklärung, was wir erreichen wollen und warum wir das tun.
%
Danach stellen wir eine Art, die Dinge zu tun vor.
%
Mit den hier vorgestellten Grundlegenden Gedanken, dem Manifest und den Entwurfsmustern legen wir eine Basis für weXelwirken aus, auf der sich dann vieles entwickeln kann.
%
Wichtig ist hierbei immer, die angebotenen Entwurfsmuster und Gedanken als potentielle Hilfen zu sehen. 
%
Wir wollen damit keine Richtlinien oder enge Grenzen setzen.



Unsere laufenden Projekte stellen wir im Kapitel Projekte vor.



Das darauffolgende Kapitel spricht die geplante Entwicklung an.
%
Hier haben wir keinen allgemeingültigen Konsens.
%
Eine Diskussionsgrundlage zur möglichen Entwicklung aus Sicht der weXelwirker zu beschreiben war hier unsere Absicht.




Dieses Buch bedarf einer ständigen Überarbeitung, eine neue Version wird jedes Jahr erscheinen.
%
Zwischendurch sind die aktuellen Texte und Ideen auf der Homepage zu finden: http://weXelwirken.net/buch
%
Bitte schauen Sie unter\\ http://weXelwirken.net/buch nach, ob Sie die aktuelle Version lesen.
%
Im Moment lesen Sie die Version 0.1 vom 13.04.2012 - laden Sie bitte die aktuellste Ausgabe herunter.
%
Wenn Sie Anregungen für Verbesserungen oder ähnliches haben, schreiben Sie an buch@weXelwirken.net .
