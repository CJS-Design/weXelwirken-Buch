\chapter{Eine Art, die Dinge zu tun}
  \section{Sind wir eine Sekte?}
Zusammenfassung, wie wir auf diese Richtung kommen, Workshops, Entwicklung usw.
%
Wir sind alle verbunden (Quantenphysik) und wirken aufeinander - weXelwirken eben.
%
Diese Verbindung löst das darwinistische Bild auf - Kein "`Jeder gegen Jeden"', sondern "`Jeder für alle"' und "`Alle für einen"'.
%
Eine Handlung hat Einfluß auf deine Umgebung, nicht nur auf die die direkte, mindestens noch auf die indirekte (2. Grad) wahrscheinlich aber auch darüber hinaus (Wie Wellen im Wasser, nach außen schwächer werdend).
%
Rückkopplungen werden kommen - Sende gutes Aus, und du wirst gutes erhalten? (Quelle?)
  \section{Manifest}
Wir sind weXelwirken. Sprich darüber!
  \section{Grundlegende weXelwirken Entwurfsmuster}
  Entwurfsmuster? Was zur Hölle ist denn das nun wieder?
    \subsection{Fokus}
Problem: Du musst /willst alles selber machen.
%
Lösung: Gib Aufgaben ab. Überwache die Ausführung in einem sinnvollen Maß.
%
Ausformulierung: Blablablubb
    \subsection{Starke Knoten}
Jeder ist ein Knoten, und jeder soll ein starker Knoten sein.
%
Du bist kein starker Knoten?
%
Lass dir helfen dich zu einem zu entwickeln (Du musst dich entwickeln, andere können nur Wege zeigen).
    \subsection{Gemeinschaft}
Dein Netzwerk kann Probleme für dich lösen, wenn du es um Hilfe fragst.
%
Siehe Kommunikation. Schwarmintelligenz.
    \subsection{Atmosphäre}
Atmosphäre beschreibt das gesamte drumherum.
%
Vertrauen, Offenheit, Authentizität, Reflektion, Wertschätzung.
    \subsection{Antimuster}
Muster sind toll.
%
Aber Muster sind Muster, und das Leben verhält sich nicht immer musterhaft.
%
Muster sind Lianen im Dschungel des Lebens. Manchmal geht man besser auf dem Boden.
  \section{Weitere Entwurfsmuster}
     \subsection{Kommunikation}
Persönliche Kommunikation ist trotz Videochat eine sehr wichtige Form des Austausches.
%
Die zusätzlich übermittelten Metainformationen sorgen für Gemeinschaftsgefühl und persönliche Bindung.
     \subsection{Dienstag}
Alle Sitzungen sind Dienstags.
%
Wir rufen den Dienstag zum allgemeinen Sitzungstag aus.
%
Was dagegen? Sprich mit uns darüber - am Dienstag!    