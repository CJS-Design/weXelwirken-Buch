\chapter{Eine Art, die Dinge zu tun}
  \section{\textcolor{red}{Ziele / Unsere Leidenschaft / Warum wir brennen / Unsere Begeisterung / Chaka}}
Jeder Mensch soll ein starker Knoten sein. Ein Knoten ist verbunden mit anderen Knoten - wie in einem Fischnetz. Wenn ein Knoten sich auflöst, flutschen die Fische raus.
%
In dieser Verbindung liegt Stärke.
%
Wenn der Mensch selbst stark ist und die Begegnungen mit den anderen Menschen auf einer Ebene stattfindet, dann ist die Grundlage für ein glückliches Leben und Arbeiten geschaffen.
    \subsection{Starke Persönlichkeiten - wie definieren wir das?}
Wie kannst du anderen trauen, wenn du dir selbst nicht traust?\\
Wie kannst du anderen helfen, wenn du dir nicht helfen lässt?\\
Wie kannst du andere lieben, wenn du dich selbst nicht liebst?\\

Unsere Inspiration hierzu haben wir aus der Antrittsrede von Nelson Mandela.
{\textcolor{red} {Am besten schaut ihr euch den Film einfach mal an:}



      \subsubsection {\textcolor{red}{Eigenverantwortung -Selbstverantwortung}
Wir haben nicht das Allheilmittel zum "Glücklichsein" erfunden. Aber wir möchten unsere direktes Umfeld etwas besser machen.

Nicht jeder Mensch kann heute Selbstverantwortlich leben. Wir werden oft von äußeren Lebensumständen beeinflusst und müssen manchmal mit Gegebenheiten umgehen können. Z.B.: Eine Alleinerziehende Mutter muss arbeiten gehen - damit das Kind zu Hause etwas zu Essen hat. 

Andere würden aber auch sagen: "Ich muss mich selbstständig machen, weil ich hier meine Arbeitszeit flexibel einteilen kann."
%
Selbstverantwortung im Sinne von weXelwirken ein erstrebenswerter Zustand.


Unserer Erfahrung nach sind es aber oft nur Ausreden, um die Verantwortung zu übernehmen und die Entscheidung selbstverantwortlich und zufrieden arbeiten zu können nicht treffen zu müssen! Wir zeigen euch gern wie Ihr moderne Arbeits- und Lebenkonzepte anwenden könnt ohne auf das "Sicherheitsgefühl" verzichten zu müssen - wenn ihr überhaupt sowas benötigt!
/begin{quote}Zitat Götz Werner: "Wer will findet Wege, wer nicht will findet Gründe"/begin{quote}
%
Wir meinen damit, das eigene Leben im Rahmen der tatsächlichen Möglichkeiten\footnote{Die tatsächlichen Möglichkeiten unterscheiden sich oft sehr stark von den empfundenen, wahrgenommenen oder offensichtlichen Möglichkeiten} zu gestalten.
%
Agieren, statt reagieren.


%
Seine Leidenschaft (engl: Passion) zu finden und zu leben. \footnote{siehe http://www.ted.com/talks/larry\_smith\_why\_you\_will\_fail\_to\_have\_a\_great\_career.html} 
%
Selbstverantwortung ohne Verzicht auf Miteinander, auf Solidarität und Gemeinschaft.
%

      \subsubsection{Selbsthilfe}
Nach der Entscheidung selbstverantwortlich zu handeln, kommt das TUN!
Wir meinen Selbsthilfe nicht im Sinne von "`Ich muss alles alleine machen"' und nicht im rechtlichen Sinne von "`Ich darf mir selbst helfen, wo der Staat nicht rechtzeitig eintrifft."'
%
Sondern im Sinne von "`Ich erkenne die Stellen, an denen ich Hilfe brauche.
%
Ich weiß, wo ich Sie bekomme.
%
Ich bin mir meiner Schwächen bewusst und stark genug, mir helfen zu lassen."'
%
Selbsthilfe im Sinne von gesundem Selbstbewusstsein, im Sinne von "`die Umwelt so kennen, dass ich mir selbst helfen kann, in möglichst jedem Fall"'.
%
Selbsthilfe als Einstellung, zu wissen, wie man Kartoffeln anpflanzt.
%
Selbsthilfe als Kompetenz, Grundlagen zu verstehen und zu beherrschen.

    \subsection{Starke Gemeinschaft}
Gemeinsam statt einsam.\\
Zusammen sind wir stark.\\
Mehr als die Summe der Teile.\\
      \subsubsection{Gemeinsam Leben}
Immer wenn ein Satz mit den Worten begonnen wird "`Bin ich die/der einzige..."' ist der Satz nicht wahr.
%
In der Gemeinschaft erkennst du Gleichgesinnte, du kannst deine Themen diskutieren und dazulernen.
      \subsubsection{Gemeinsam Wirtschaften}
Genau so wie wir nicht alleine leben müssen, müssen wir auch nicht alleine wirtschaften.
%
Gemeinsam wirtschaften ist existenziell für den Aufbau von Sicherheit und Beständigkeit.
%
Zusammen erreichen wir in kürzerer Zeit mehr als alleine.
 
  \section{weXelwirken Entwurfsmuster}
weXelwirken  verwendet  sogenannte  Entwurfsmuster (engl.:  „Design  Pattern“),  um  bestimmte Phänomene zur beschreiben, die uns im Alltag begegnen.
%
Sie sind nach einem bestimmten Schema aufgebaut  und  bestehen  meist  aus  einem  Namen,  einem  charakterisierenden  Bild,  einer Problemstellung und einem Lösungsansatz.
%
Die von uns identifizierten Muster schweben nicht einfach so im luftleeren Raum, sondern sind immer eingebettet  in  unseren  Lebenskontext.
%
 In  erster  Linie  betrachten  wir  unser  Konzept  aus  dem Blickwickel als Selbständige und Freiberufler.
%
Aber wir haben inzwischen die Erfahrung gemacht, daß wir durch unsere Art die Dinge zu tun einen Raum erschaffen, der auch viele Menschen ohne diesen 
Hintergrund sehr anspricht.
%
 Gerade freischaffende Kreative aller Art, aber auch Schüler, Studenten und „normale“ Angestellt entdecken das weXelwirken Umfeld für sich als das, was die Soziologie als „dritten Ort“ zwischen Wohnung und Arbeit beschreibt.
%
Daher fassen wir den Begriff der „Selbständigen Arbeit“ sehr weit und beziehen explizit auch Aktivitäten mit ein, die nicht dem primären Ziel der Erwerbsarbeit dienen.
%
Wir begreifen weXelwirken als konzeptuellen Raum der freien Entfaltung und Selbstverwirklichung.
%
 Die monetäre Verwertbarkeit unserer  Arbeitsergebnisse  ist  sicher  ein  interessanter  Aspekt  und  Antrieb,  aber  sie  darf  kein 
Ausschlusskriterium für all jene darstellen, deren Einkommen schon durch eine andere Art der Arbeit gedeckt ist.
%
Die hier beschrieben Muster sind zwar aus unserer Sichtweise als Freiberufler entstanden, aber problemlos auch auf andere Formen der selbstbestimmten Aktivität anwendbar. 
%
Umgekehrt bereichert uns als Freiberufler die gemeinsame Nutzung von Strukturen, der Austausch von Ideen und die Arbeit mit anderen Menschen, die nicht der allgemeine Verwertungslogik ausgesetzt sind.
%
In diesem Raum der Möglichkeiten entstehen kreative Ansätze, Sichtweisen und Lösungen, die einen wichtigen Beitrag zum geschäftlichen Erfolg liefern und uns als Gesellschaft voranbringen können.


\subsection{Grundlegende weXelwirken Muster}
    \subsubsection{Starke Knoten}
Jeder ist ein Knoten, und jeder soll ein starker Knoten sein.
%
Du bist kein starker Knoten?
%
Lass dir helfen dich zu einem zu entwickeln (Du musst dich entwickeln, andere können nur Wege zeigen).
    \subsubsection{Fokus}
\begin{em} Du musst /willst alles selber machen.
%
40 bis 60 Prozent der Zeit verbringst du als Selbständige mit Dingen, die nicht im eigenen Aufgabenbereich liegen.
\end{em}


Gib Aufgaben ab.
%
Überwache die Ausführung in einem sinnvollen Maß.
%
Identifiziere Deinen Fokus und delegiere alle anderen Arbeiten konsequent an Menschen, deren Fokus das ist.
     \subsubsection{Gemeinschaft}
\begin{em} Du fühlst Dich allein mit Deinen Ansichten und Einstellungen und fragst Dich, ob Du der einzige bist, 
der so denkt. \end{em}


Umgib Dich mit Menschen, die Deine Werte und Vorstellungen teilen.
%
Tausch Dich mit ihnen aus und schaffe Strukturen in denen dieser Austausch stattfinden kann.

\subsubsection{Projekte}
\begin{em}
Du teilst Deine Ressourcen vertrauensvoll mit anderen und konzentrierst Dich mit Hilfe Deines 
Netzwerks auf Deinen Fokus. Du bist in erfolgreich und effizient, aber eine gesellschaftliche Wirkung 
will sich nicht so recht einstellen.
\end{em}



Entwickle mit anderen aus Deinem Umfeld ein gemeinsame Projekte, die in Deinen gesellschaftlichen 
Kontext  hineinwirken.  Mach  aus  Deinem  Büro  einen  CoWorkingSpace,  veranstalte  regelmäßige 
Diskussionsrunden  zu  spannende  Themen  aus  Technik,  Politik  und  Gesellschaft.  Gründe  einen 
Selbsthilfe-Club für Fahrradbastler.

    
    \subsubsection{Antimuster}
Muster sind toll.
%
Aber Muster sind Muster, und das Leben verhält sich nicht immer musterhaft.
%
Muster sind Lianen im Dschungel des Lebens. Manchmal geht man besser auf dem Boden.
  \subsection{Weitere Entwurfsmuster}

\subsubsection{Teilen}
\begin{em}
Du willst ein Büro, hast aber keine Räume.
%
Du braucht bestimmt Werkzeuge, aber kannst sie Dir nicht leisten.
%
Etwas funktioniert nicht, aber Du weißt nicht, warum.
\end{em}



Teile Dir möglichst viele Ressourcen mit anderen.
%
Egal ob Wissen, Ausrüstung oder Räumlichkeiten, selten braucht man diese Dinge wirklich exklusiv für sich alleine.
%
Gemeinsam bist Du stärker, schlauer, reicher.
%
Eine Vorraussetzung für Teilen ist Atmosphäre.

     \subsubsection{Kommunikation}
Persönliche Kommunikation ist trotz Videochat eine sehr wichtige Form des Austausches.
%
Die zusätzlich übermittelten Metainformationen sorgen für Gemeinschaftsgefühl und persönliche Bindung.

   \subsubsection{Atmosphäre}
Atmosphäre beschreibt das gesamte drumherum.
%
Vertrauen, Offenheit, Authentizität, Reflektion, Wertschätzung.
     \subsubsection{Dienstag}
Alle Sitzungen sind Dienstags.
%
Wir rufen den Dienstag zum allgemeinen Sitzungstag aus.
%
Was dagegen? Sprich mit uns darüber - am Dienstag!

\subsubsection{Netzwerk}
\begin{em}
Um Arbeiten, die nicht im eigenen Fokus liegen, weitergeben zu können braucht man ein möglichst 
weitläufiges Netzwerk aus kompetenten Kollegen.
\end{em}



Die fünf Menschen, mit denen Du am häufigsten zusammen arbeitest, kennen jeweils auch wieder fünf 
Leute, die fünf Leute kennen.
%
Dein Netzwerk ist schon da, Du mußt es nur benutzen.
%
Dein Netzwerk kann Probleme für dich lösen, wenn du es um Hilfe fragst.
%
Siehe Kommunikation. Schwarmintelligenz.
   


\subsubsection{Vertrauen}
\begin{em}
Woher weiß man, daß der Kollege wirklich kann, was er von sich behauptet? Wer sagt einem, daß sich 
Kunden an die getroffenen Absprachen halten? Wirst Du auch morgen noch genug Aufträge haben, um 
Dein Einkommen zu sichern?
\end{em}



Gib den Dingen vertrauensvoll den Raum, den sie brauchen, um sich zu entfalten. Übermäßige 
Kontrolle wird Dich und Deine Welt so einengen, daß Du weder erfolgreich noch glücklich sein wirst.

\subsubsection{Selbständige Arbeit}

\subsubsection{Freundeskreis}

\subsubsection{Gesellschaft}


\subsection{Weiter Mustersets}

\subsubsection{Design Patters for Hackerspaces}

\subsubsection{A Pattern Language}