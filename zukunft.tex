\chapter{Entdecke die Möglichkeiten / ...und morgen?}

  \section{2012}
\subsection{Rechtliches und Finanzielles, unser Business-Konzept}

Jede Idee braucht ein finanzielles und rechtliches Rückrat. 
Derzeit planen wir die Gründung einer Genossenschaft und / oder eines Vereins.

Die Zielgruppe:
Unternehmer u Freiberufler mit max. 5 MA
Gründer
Angestellte, die gern flexibel und ortsunabhängig arbeiten 


DAS KONZEPT
weXelwirken baut auf 4 Säulen
1. Netzwerk
2. CoWorkingSpace
3. CoConsumption 
4. Fairness 

Grundsätze:
Jeder Space oder jedes einzelne Mitglied handelt selbstverantwortlich und unabhängig von den anderen Mitgliedern - Dezentral!



1. Netzwerk
Hier sollen diverse Gewerke untereinander verknüpft werden um das Geschäftsleben der einzelnen zu erleichtern. Darüber hinaus sollen gemeinsam größere Aufträge abgewickelt werden. Der Auftragsnehmer kann wXw sein oder einer der Mitglieder direkt.
Sollte wXw als Auftragsnehmer fungieren ist immer ein hauptverantwortliche Projektleiter zu benennen. Die Zustimmung der Auftragsannahme ist durch den Vorstand zu Erfolgen.
wXw sollte Platz für Themen u Spinnereien schaffen außerhalb des Business und Tagesgeschäft als Selbstständiger.

1.1 Kommunikation 

Jeder Space sollte mind. einen web 2.0. Chanel seiner Wahl betreiben um seinen "`Dunstkreis"' zu informiert und ggf. neue Mitglieder zu generieren. Alternativ können auch andere Kommunikationswege gewählt werden z.B. Newsletter.
Für die Allgemeinheit stehen bereits folgende Kommunikationskanäle zur Verfügung:
Facebook Fanpage "`Wexelwirken …"'
Twitter: weXelwirken
bei Xing sind wir als Unternehmen registriert
Interner E-Mail Verteiler, hier kann jeder an alle Teilnehmer von wXw eine Mail senden 


1.2 Förderung der Gemeinschaft
erfolgt am Einfachsten bei Veranstaltungen. 
Regional werden diese durch die CoWorkingSpace Betreiber organisiert. Unsere Empfehlung: mind. 6 x jährlich 
Die Themen und die Häufigkeit der Termine sind den Betreibern offen gestellt. Wir empfehlen hier:
- dritte einzubinden um den Aufwand für den Space-Betreiber zu reduzieren
- mind. 1 Monat Vorlauf, besser sind 2 Monate
- nutzen sie alle Ihnen zur Verfügung stehende Kommunikationskanäle um auf die Veranstaltung aufmerksam zu machen


2. CoWorkingSpace
Neben den virtuellen und persönlichen Knotenpunkten aus dem Netzwerk, müssen auch reale Standorte und Anlaufpunkte geschaffen werden, die den Austausch und die Zusammenarbeit fördern sollten.
Aus organisatorischen Gründen muss jeder Space-Betreiber mind. eine Mitgliedschaft erwerben.
Die Bürobetreiber haben natürlich einen vermehrten Aufwand, bedingt durch die Organisation und die Verantwortung. Als Gegenzug erhalten die Bürobetreiber einen höheren Einblick und Mitspracherecht in die Gesamtorganisation von weXelwirken.
//
Dies geschieht durch:
Wöchentliche Telefonkonferenzen
Regelmäßige persönliche treffen (jeder Space sollte mind. 1 Mal jährlich eine Veranstaltung organisieren, bei der sich die einzelnen Space-Betreiber treffen und austauschen können. Die Teilnahme sollte an mindestens 4 Treffen / jährlich erfolgen.
//
Darüber hinaus ergeben sich weitere Vorteile für die Space-Betreiber: 
Auswahl der Mieter / Teilnehmer und somit Beeinflussung des eigenen Teams
Nutzung der Struktur und des Bekanntheitsgrades von WeXelwirken in der Außenkommunikation (etablierte Dachmarke)
Günstigere Preise im Einkauf (durch Mengenrabatte von wXw gesamt), Werbung (Werbegemeinschaften) und weitere gemeinschaftliche genutzte Objekte
Kontakte zu potenziellen Kunden innerhalb von wXw Bundesweit



3. CoConsumption
Einer klassischen Genossenschaft gleichgestellt, sollen die Mitglieder Objekte, Ressourcen, Werbemöglichkeiten gemeinsam nutzen können.
Definition Wikipedia:
is used to describe an economic model based on sharing, swapping, bartering, trading or renting access to products as opposed to ownership
Organisation?
Zuständigkeiten ggf. nach Bereichen aufteilen

4. Fairness
die Marke weXelwirken soll nach außen kommunizieren, dass die Mitgliedschaft in diesem Verein / Genossenschaft / Netzwerk für bestimmte Werte und eine bestimmte Zusammenarbeit steht. 
Siehe Muster: Vertrauen (link)

Rechtliches und Finanzielles
Jeder Space-Betreiber arbeitet eigenverantwortlich und auf eigene Haftung. 
Um die Vorteile einer "`Dachmarke"' zu nutzen ist jedoch eine Mitgliedschaft in der Genossenschaft notwendig.

Wer kann Mitglied werden?
Potenzielle Mitglieder müssen mind. bereits 2 Genossen kennen, die sich für ihn/sie positiv aussprechen würden und die Patenschaft für das 1. Jahr übernehmen können.

Es wird zunächst eine Probe-Mitgliedschaft von 12 Monaten für 50,00 EUR ?? abgeschlossen. Nach dieser Zeit wird bei einer der Hauptversammlungen abgestimmt, ob eine definitive Mitgliedschaft erfolgen kann. Hier müssen mind. 50 Prozent der Genossen anwesend sein.

Jeder Genosse übernimmt eine Tätigkeit (5 Stunden monatl.), die dem Allgemeinwohl dient. Offen wären noch:
- P/R
- Betreuung Homepage
- Telefonzentrale für allgemeine Auskünfte
- Vorträge bei diversen Institutionen
- Grafische Arbeiten
- Steuerberatung / Kasse
…
Kosten einer Mitgliedschaft:
- Aufnahmegebühr: 0,00 EUR (durch die Probe-Mitgliedschaft gedeckt)
- Jahresgebühr … EUR (Verwendung wird demokratisch beschlossen)
- Kauf eines Anteils, derzeitiger Wert siehe unten.
Die Genossenschaft hat das Ziel +/- 0 zu wirtschaften.
Alle Gewinne werden an die Mitglieder anteilsmäßig zurückgeführt.
Anfallende Kosten werden anteilsmäßig aufgeteilt (falls durch die Jahresgebühr nicht gedeckt, erfolgt eine Abstimmung)

Bsp.: Berechnung der Anteile:
Gründung der Genossenschaft: 3000,00 EUR
Branko, Christopher und Eugenia beteiligen sich jeweils mit 1.000,00 EUR
Jeder hat 10 Anteile, zu je einem Wert von 100,00 EUR.
Abverkauf der Anteile erfolgt durch Branko, Christopher und Eugenia abwechselnd.


Rabatte für weXelwirker-Genossen bei:
e.allerdings – Vertrieb und Marketing, halle westf.
Vertriebsunterstützung, Marktrecherche, zielgruppengerechte Neukundengewinnung, Web 2.0. Redaktion
5 Prozent auf alle Dienstleistungen
Branko Čanak, paderborn
Christopher Schmidhofer, Wankheim
…


 


Definition Genossenschaft
Eine Genossenschaft hat in der Regel drei Organe: Vorstand, Aufsichtsrat und die Generalversammlung. Es müssen zwei Vorstandsmitglieder (\S 24. GenG) und drei Aufsichtsratsmitglieder (\S 36 GenG) gewählt werden. Bei Genossenschaften mit nicht mehr als 20 Mitgliedern kann der Vorstand auch aus nur einem Mitglied bestehen und auf einen Aufsichtsrat verzichten. In diesem Fall nimmt die Generalversammlung die Funktionen des Aufsichtsrats wahr.
Die Mitglieder der Genossenschaft sind ähnlich wie die Gesellschafter einer AG. Sie kaufen im Rahmen der Mitgliedschaft Anteile an der Genossenschaft und stellen auf diese Weise Eigenkapital zur Verfügung.
Die wichtigsten Schritte im Rahmen der Gründung einer Genossenschaft:
Kontaktaufnahme mit dem Genossenschaftsverband/Gründungsprüfer
Erarbeitung eines Gründungskonzeptes
Genossenschaftliche Gründungsversammlung
Gründungsprüfung
Anmeldung zur Eintragung in das Genossenschaftsregister

  \section{2013}
\subsection{Rechtsform Verein}
Eine Vereinsform für weXelwirken wäre in mehreren Punkten vorteilhaft.
%
Im Moment haben wir einige Menschen im Dunstkreis von weXelwirken, die mitgestalten und mitdiskutieren, aber nicht CoWorker sind.
%
Einige dieser Menschen sind historisch verbunden, aber inzwischen an anderen geographischen Orten unterwegs.
%
Einige sind regelmäßig an Veranstaltungen anwesend.
%
Diese Menschen einzubinden und mit weXelwirken enger zu verbinden, dazu wäre eine Vereinsform passend.
%
Nach extern gesehen ergebn sich für den Verein verschiedene Möglichkeiten, Kosten zu sparen.
%
In den Amtsblättern dürfen Vereine meist Veranstaltungen und ähnliches kostenlos ankündigen, Gemeindeeinrichtungen dürfen kostenlos oder vergünstigt mitbenutzt werden.
%
Die Räume könnten als Vereinsstätten deklariert und verwaltet werden (genauere Infos einholen!)
%
Diese Idee wurde auch von unserem Mentor Declan Kennedy aufgebracht.



Damit wäre eine Basis für die Verwaltung der Veranstaltungen, der Räumlichkeiten geschaffen, sowie eine Bindung an weXelwirken auch für Nicht CoWorker geschaffen.



Die gewünschte wirtschaftliche Kooperation benötigt eine eigene Form.
%
Hier ist der Gedanke an eine Genossenschaft immer noch verlockend.
%
Durch die Trennung in zwei eigenständig agierende Rechtsformen kann die Verschiedenheit der Ausrichtung, durch die gemeinsame Basis (wie in diesem Buch beschrieben) die Verknüpfung dargestellt werden.
%
Die Genossenschaft hat als einzige Rechtsform die Möglichkeit der flexiblen und dynamischen Mitgliederverwaltung.
%
Die meisten anderen Rechtsformen benötigen dafür Änderungen an Gesellschafterverträgen oder mehr Aufwand.
%
Die Frage, ob wir eine gemeinsame Rechtsform benötigen (Notwendigkeit) oder einfach nur schick finden, ist hier die falsche.
%
Es kommt darauf an, was die Interessenten an einer gemeinsamen Rechtsform interessant genug finden, den entsprechenden Aufwand da rein zu stecken.
%
Hier ist sicherlich noch Diskussionbedarf.
%
Es bleibt festzuhalten, dass in einer solchen gemeinsamen Rechtsform sicherlich nicht alle weXelwirker vertreten sein werden.
  \section{2014}  
  \section{2015}
  \section{2016}
  \section{2017}
  \section{2018}
  \section{2019}
\subsection{Verbreitung, Außenauftritt}
Hinaus in die Welt
Um weXelwirken als Idee zu verbreiten und zu fördern, verfolgen wir eine dreiteile Strategie. In 
Vorträgen machen wir auf weXelwirken aufmerksam, in Workshops vermitteln das Konzept und über 
Qualifizierungen sichern wir die Qualität unserer Arbeit.
Vorträge
Vorträge über weXelwirken können in den unterschiedlichsten Kontexten stattfinden. Vom kleinen 
Smalltalk am Rande eines Sektempfangs bis hin zur abendfüllenden Veranstaltung mit anschließender 
Diskussion ist alles denkbar. Das Ziel dieser Vorträge ist immer, Menschen so für weXelwirken zu 
begeistern, daß sie selber nach diesem Konzept tätig werden. Wichtig ist ein Gespür für die zur 
Verfügung stehende Zeit und die damit verbundene Aufmerksamkeit des Zuhörers zu entwickeln. Aus 
unserer Erfahrung ergibt sich oft das im folgenden beschriebene Raster. In den Anlagen zu diesem 
Konzept werden wir im Laufe der Zeit Vorlagen für die verschiedenen Vortragssituationen sammeln.
1 Minute
Eine Minute ist die Zeit, die einem in einer klassischen Vorstellungssituation zur Verfügung steht. Sie 
kann Auftakt zu einem Gespräch sein, in dem man seine Interessen und Ideen noch näher darstellen 
wird. Befindet man sich in einer größeren Gruppe, ist das aber auch die Zeit, die man maximal in einer 
ersten Vorstellungsrunde das Wort ergreift.
Wichtig in dieser Zeitkategorie ist, kurze, prägnante Kernmaussagen zu äußern und neugierig auf mehr 
zu machen, ohne Begründungen zu liefern oder in Details einzusteigen. 
0 52 51/50 660 48
info@wexelwirken.net
17
5 Minuten
Fünf Minuten sind ein Zeitrahmen, die einem in typischen Smalltalk-Situationen zur Verfügung stehen. 
Aber auch als „Werbung“ für einen längeren Vortrag, wie zum Beispiel auf einem BarCamp, ist diese 
Zeitschiene geeignet.
Wichtig ist hier, wie in „1 Minute“ ungezwungen mit einer Kernaussage zu beginnen und schon etwas 
über die Hintergründe sprechen, ohne dabei aber auf weitere Rückfragen eingehen zu müssen.
15 Minuten
In fünfzehn Minuten kann man schon eine vollständige Argumentationslinie durchlaufen und auch auf 
entsprechende Rückfragen eingehen. Diese Zeitkategorie eignet zum Beispiel sich für Lightning Talks 
oder Impulsvorträge mit anschließender Diskussion.
Hier ist wichtig am Ende ein Thema schlüssig und vollständig dargestellt zu haben. Daher lieber 
weniger und dafür vollständig vermitteln, als zu viel und in sich nicht geschlossen.
1 Stunde
Wenn einem eine Stunde oder mehr zu Verfügung stehen, kann man in dieser Zeit das Konzept in seiner 
ganzen Breite vorstellen auf einzelne Aspekte detailliert eingehen. Dieser Zeitraum eignete sich für 
klassische Vortragssituationen, wie man sie bei Abendveranstaltungen oder auf Tagungen vorfindet.
In dieser Zeitkategorie ist es wichtig über die länge des Vortrags die Zuhörer nicht zu verlieren, 
sondern aktiv mit einzubeziehen. Dies kann dadurch geschehen, daß man zu Zwischenfragen einlädt, 
einen Diskussionsteil mit einplant oder die Teilnehmer einen thematisch geeigneten Selbstversuch 
durchführen läßt. 
0 52 51/50 660 48
info@wexelwirken.net
18
Workshops
In unseren Workshops vermitteln wir, was das weXelwirken Konzept beinhaltet und wie man es konkret 
auf sich und seine Umwelt anwendet. Zunächst gibt es drei aufeinander aufbauende Teile, die aber in 
Zukunft durch neu Bausteine erweitert werden können.
WXW Basics
In diesem Workshop vermitteln wir die konzeptuellen Grundlagen, die Arbeit mit den weXelwirken 
Mustern und wie man diese auf die eigene Situation anwendet. Folgende typische Fragen werden in 
diesem Baustein beantwortet:
•
Was ist weXelwirken?
•
Wie arbeitet man mit Entwurfsmustern?
•
Welche weXelwirken Muster gibt es?
•
Wie wendet man diese Muster im eignen Kontext an?
WXW Projekte
In diesem Workshop unterstützen wir Teilnehmer für ihre Situation geeignete Projekte zu entwickeln. 
Folgende typische Fragen werden in diesem Baustein beantwortet:
•
Was sind weXelwirken Projekte?
•
Wie können Projekte mein Arbeitsumfeld bereichern?
•
Welche Projekte gibt es schon und wie funktionieren sie?
0 52 51/50 660 48
info@wexelwirken.net
19
WXW Spaces
Als ein bisher sehr häufig auf tauchendes Projekt sind CoWorking- und HackerSpaces für unsere Arbeit 
mit dem weXelwirken Konzept von besonderer Bedeutung. Sie schaffen eine räumliche Struktur, in der 
sich vielfältige Aktivitäten entwickeln können und geben einen lokalen Anknüpfungspunkt für Aktive 
und Interessierte. Folgende typische Fragen werden in diesem Baustein beantwortet:
•
Was sind CoWorkingSpaces?
•
Was ist beim Aufbau eines Spaces zu beachten?
•
Wie betreibt man einen Space nachhaltig und zum Nutzen aller Beteiligten?
Qualifizierung
Langfristig möchten wir weXelwirken zu einer sich selbstverstärkenden Struktur von hoher Qualität 
entwickeln. Zwei Instrumente dafür sind ein Qualitätssiegel und eine Fortbildung zum WXW Trainer.
WXW Siegel
Menschen, die mit dem weXelwirken Konzept arbeiten, handeln verantwortungsvoll und gehen bewußt 
mit ihren Möglichkeiten, Ressourcen und ihrer Umgebung um. Sie sind so erfolgreicher und liefern 
ihren Kunden hochwertigere Produkte und Dienstleistungen, als andere. Dies können sie über den 
Erwerb des WXW Siegels nach außen dokumentieren. Vorraussetzungen für dieses Qualitätssiegel sind:
•
Teilnahme am WXW Basics Workshop
•
Anwendung des weXelwirken Konzepts in der täglichen Arbeit
•
Jährliche Supervisionen zur Reflexion des eigenen Handels   
0 52 51/50 660 48
info@wexelwirken.net
20
WXW Trainer
Um das weXelwirken Konzept zu verbreiten, bieten wir eine Fortbildung zum WXW Trainer an. 
Teilnehmer dieses Fortbildung dürfen ihrerseits wieder weXelwirken Workshops veranstalten und in die 
Nutzung des Konzepts einführen. Vorraussetzungen dafür sind:
•
Teilnahme an allen Workshop-Typen
•
Erwerb des WXW Siegels
•
Jährliche Teilnahme an einem Trainertreffen
  \section{2020}
  \section{2021}
  \section{2022}
100 weXelwirken Standorte international und über 1000 weXelwirker, die gemeinsam an einer begeisterten Menschheit bauen.
