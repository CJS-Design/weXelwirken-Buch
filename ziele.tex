
    \subsection{Das Leben, das Universum und der ganze Rest}
42.
    \subsection{Gemeinsam Wirtschaften}
      \subsubsection{Kooperation}
Kooperation statt Konkurrenz.
%
In der Kooperation liegt die Chance.
%
Wie kann diese Kooperation angeregt werden?
%
Brauchen wir einen (rechtlichen) Rahmen dafür?
%
Wie kann dieser gestaltet sein? Genossenschaft? Verein?
      \subsubsection{CoWorking}
CoWorking eignet sich für viele Menschen.
%
Hauptsächlich werden CoWorking-Spaces ("`Zusammenarbeitsräume"') von Selbstständigen und Freiberuflern genutzt.
%
Aber auch Künstler und Studenten nutzen die kreative Arbeitsatmosphäre gerne.
%
Für Angestellte kann es eine Alternative zum Home Office sein.
%
Durch den Austausch ergibt sich ein Blick über den eigenen Tellerrand.
%
Laut Umfragen von http://www.deskmag.com/ steigerten 48\% der CoWorker ihren Verdienst nach Einstieg in einen CoWorking-Space.
%
Das ist doch ein deutliches Zeichen.



CoWorking ist eine Form von gelebtem weXelwirken.
%
CoWorking schafft Freiräume zur Entwicklung.
%
Die anwesende Gemeinschaft wird durch räumliche Nähe gestärkt.
%
Beim "`nebeneinander arbeiten"' ergeben sich Synergien und Projekte, vorteilhaft für alle.
      \subsubsection{Ko-Konsum}
Teilen statt Besitzen.
%
Brauchen wir die Bohrmaschine oder das Loch?



















 \section{Sind wir eine Sekte?}
Zusammenfassung, wie wir auf diese Richtung kommen, Workshops, Entwicklung usw.
%
Wir sind alle verbunden (Quantenphysik) und wirken aufeinander - weXelwirken eben.
%
Diese Verbindung löst das darwinistische Bild auf - Kein "`Jeder gegen Jeden"', sondern "`Jeder für alle"' und "`Alle für einen"'.
%
Eine Handlung hat Einfluß auf deine Umgebung, nicht nur auf die die direkte, mindestens noch auf die indirekte (2. Grad) wahrscheinlich aber auch darüber hinaus (Wie Wellen im Wasser, nach außen schwächer werdend).
%
Rückkopplungen werden kommen - Sende gutes Aus, und du wirst gutes erhalten? (Quelle?)
