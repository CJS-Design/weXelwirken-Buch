\chapter{Ziele}
  \section{Starke Knoten}
Wie kannst du anderen trauen, wenn du dir selbst nicht traust?\\
Wie kannst du anderen helfen, wenn du dir nicht helfen lässt?\\
Wie kannst du andere lieben, wenn du dich selbst nicht liebst?\\
    \subsection{Selbstverantwortung}
Nicht jeder Mensch kann heute Selbstverantwortlich leben.
%
Oft werden Menschen von äußeren Umständen gezwungen, z.B. von solchen, die wir in unserer verwöhnten westlichen Welt gar nicht begreifen.
%
Nichtsdestotrotz ist Selbstverantwortung im Sinne von weXelwirken ein erstrebenswerter Zustand.
%
Wir meinen damit, das eigene Leben im Rahmen der tatsächlichen Möglichkeiten\footnote{Die tatsächlichen Möglichkeiten unterscheiden sich oft sehr stark von den empfundenen oder offensichtlichen Möglichkeiten} zu gestalten.
%
Agieren, statt reagieren.
%
Seine Leidenschaft (engl: Passion) zu finden und zu leben. \footnote{siehe http://www.ted.com/talks/larry\_smith\_why\_you\_will\_fail\_to\_have\_a\_great\_career.html} 
%
Selbstverantwortung ohne Verzicht auf Miteinander, auf Solidarität und Gemeinschaft.
%
NEU FORMULIEREN. Es bedeutet, zu akzeptieren wie du selbst bist, selbst zu entscheiden wie du sein möchtest und daran zu arbeiten, du zu bleiben und dein Traum zu werden.
    \subsection{Selbsthilfe}
NEU FORMULIEREN. Selbsthilfe nicht im Sinne von "`Ich muss alles alleine machen"'.
%
Nicht im rechtlichen Sinne von "`Ich darf mir selbst helfen, wo der Staat nicht rechtzeitig eintrifft."'
%
Sondern im Sinne von "`Ich erkenne die Stellen, an denen ich Hilfe brauche.
%
Ich weiß, wo ich Sie bekomme.
%
Ich bin mir meiner Schwächen bewusst und stark genug, mir helfen zu lassen."'
%
Selbsthilfe im Sinne von gesundem Selbstbewusstsein, im Sinne von "`die Umwelt so kennen, dass ich mir selbst helfen kann, in möglichst jedem Fall"'.
%
Selbsthilfe als Einstellung, zu wissen, wie man Kartoffeln anpflanzt.
%
Selbsthilfe als Kompetenz, Grundlagen zu verstehen und zu beherrschen.
  \section{Starke Gemeinschaft}
Gemeinsam statt einsam.\\
Zusammen sind wir stark.\\
Mehr als die Summe der Teile.\\
    \subsection{Das Leben, das Universum und der ganze Rest}
42.
    \subsection{Gemeinsam Wirtschaften}
      \subsubsection{Kooperation}
Kooperation statt Konkurrenz.
%
In der Kooperation liegt die Chance.
%
Wie kann diese Kooperation angeregt werden?
%
Brauchen wir einen (rechtlichen) Rahmen dafür?
%
Wie kann dieser gestaltet sein? Genossenschaft? Verein?
      \subsubsection{CoWorking}
CoWorking eignet sich für viele Menschen.
%
Hauptsächlich werden CoWorking-Spaces ("`Zusammenarbeitsräume"') von Selbstständigen und Freiberuflern genutzt.
%
Aber auch Künstler und Studenten nutzen die kreative Arbeitsatmosphäre gerne.
%
Für Angestellte kann es eine Alternative zum Home Office sein.
%
Durch den Austausch ergibt sich ein Blick über den eigenen Tellerrand.
%
Laut Umfragen von http://www.deskmag.com/ steigerten 48\% der CoWorker ihren Verdienst nach Einstieg in einen CoWorking-Space.
%
Das ist doch ein deutliches Zeichen.



CoWorking ist eine Form von gelebtem weXelwirken.
%
CoWorking schafft Freiräume zur Entwicklung.
%
Die anwesende Gemeinschaft wird durch räumliche Nähe gestärkt.
%
Beim "`nebeneinander arbeiten"' ergeben sich Synergien und Projekte, vorteilhaft für alle.
      \subsubsection{Ko-Konsum}
Teilen statt Besitzen.
%
Brauchen wir die Bohrmaschine oder das Loch?
